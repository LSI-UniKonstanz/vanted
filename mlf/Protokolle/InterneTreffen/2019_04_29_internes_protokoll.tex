%! TeX program = lualatex
\documentclass[11pt]{article} % Schriftgrösse 12;
\usepackage[utf8]{inputenc} % Verwendung von Umlauten im tex
\usepackage[T1]{fontenc}    % Neues LaTeX Font Encoding für korrekte Darstellung von Umlauten im PDF
\usepackage[ngerman]{babel}  % Neue deutsche Rechtschreibung
\usepackage[babel]{csquotes} % Anführungszeichen leicht gemacht: \enquote{Text in Anführungszeichen}
\usepackage{amsmath}  % Verbesserungen für mathematischen Formelsatz
\usepackage{amssymb}  % Erweiterte mathematische Symbole, lädt Paket 'amsfonts'
\usepackage{amsthm}   % Theoremumgebungen
\usepackage{marvosym} % Mehr Symbole. Zum Beispiel: Euro-Zeichen, Warnhinweise
\usepackage[colorlinks=true, urlcolor=blue]{hyperref} % Hyperlinks mit blauer Schrift
\usepackage{graphicx}                                 % Einbinden von Grafiken (JPG, PNG, SVG, ...)
\usepackage{listings}                                 % Darstellung/Einbinden von Source Code
\usepackage{enumitem}
% TikZ bietet viele nützliche Bibliotheken, die wir verwenden können (einfach das Kommentarzeichen entfernen)

% Code-Highlighting soll standardmässig Java Code darstellen
\lstset{language=Java, breaklines=true, showstringspaces=false}
% Beispiel-Code für die Darstellung von Java Code
%\begin{lstlisting}
%     Hier würde der Java-Code hinkommen und entsprechend die Syntax markiert. Selbst einrücken.
%\end{lstlisting}
%ODER:
% \lstinputlisting[language=Java]{name.java}

% Formatierung der Seite damit wir mehr Platz haben
\topmargin -2cm
\textheight 24cm
\textwidth 16.0 cm
\oddsidemargin -0.1cm

%%%%%%%%%%%%%%%%%%%%%%%%%%%%%%%%%%%%%%%%%%%%%%%%%%%%%%%%%%%%%%%%%%%%%%%%%%%%%%%%%%%%%%%%%%%%%%%%%%%%%%%%%%%%%%%%


%\mathcode`,="013B
%\mathcode`.="613A
% Kopfzeile
\usepackage{datetime}

\newcommand{\names}{Team 2.2}
\newcommand{\deadline}{\formatdate{6}{7}{2018} 12:00}


\begin{document}
\thispagestyle{plain}
{\footnotesize \parindent0em
    {\sc Universität Konstanz} \hfill {\sc Softwareprojekt}\par
    {\sc Fachbereich Informatik \& Informationswissenschaft} \hfill Wintersemester 2019 \par
    \textbf{\names} \hfill \today\par
    \begin{center}
      {\Large\bf Protokoll --- Internes Treffen am \today}
    \end{center}}
  \par{\textbf{Anwesend:}} Jannik, Tobias, René, Theodor, Gordian
\subsubsection*{Inhalte:}
  \begin{itemize}
  \item Recap des Initialmeetings/der Problemstellung
    \begin{itemize}
    \item Diskussion was als erstes zu tun ist
    \item Erklärung der Grundidee von Stress Minimization und
      des Multilevel Frameworks am Whiteboard
    \item Klärung, mit welcher Art von Graphen wir arbeiten müssen
      (ungerichtet und keine Kantengewichte (?))
    \item Klärung des Beispiels mit dem Dreiecksgraph (Sierpiński triangle) aus einem der Paper
    \end{itemize}
  \item Aufteilung in (temporäre) Arbeitsgruppen
    \begin{itemize}
    \item Stress Minimization: Theodor, René, Jannik
    \item Multilevel Framework: Gordian, Tobias, Jonas
    \end{itemize}
  \item Besprechung, was für das Pflichtenheft zu tun ist
    (Arbeitsgruppen sollen sich ins jeweilige Thema einlesen,
    und den Aufwand für den Zeitplan grob abschätzen)
  \item Ausmachen eines Termins, an dem wir uns treffen, um
    das Pflichtenheft zu machen:
    \begin{itemize}
    \item {\large \bf Mittwoch 1. Mai, um 10 (vor der Bibliothek)}
    \end{itemize}
  \end{itemize}  
\end{document}
